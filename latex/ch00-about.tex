\documentclass{beamer}
\usepackage{ctex}
\usepackage{beamerthemetree}
\usepackage{graphicx}
\usepackage[autoplay,loop]{animate}
\usepackage{pgf}
\usepackage{tikz}
\usepackage{fancybox}
\usepackage{movie15}
\usepackage{verbatim}
%\usetheme{default}
\usetheme{AnnArbor}
%\usetheme{Antibes}
%\usetheme{Bergen}
%\usetheme{Berkeley}
%\usetheme{Berlin}
%\usetheme{Boadilla}
%\usetheme{CambridgeUS}
%\usetheme{Copenhagen}
%\usetheme{Darmstadt}
%\usetheme{Dresden}
%\usetheme{Frankfurt}
%\usetheme{Goettingen}
%\usetheme{Hannover}
%\usetheme{Ilmenau}
%\usetheme{JuanLesPins}
%\usetheme{Luebeck}
%\usetheme{Madrid}
%\usetheme{Malmoe}
%\usetheme{Marburg}
%\usetheme{Montpellier}
%\usetheme{PaloAlto}
%\usetheme{Pittsburgh}
%\usetheme{Rochester}
%\usetheme{Singapore}
%\usetheme{Szeged}
%\usetheme{Warsaw}
\usecolortheme{lily}
\newcommand{\colorize}[2]{{\color{#1} #2 }}
\newcommand{\items}[1]{\begin{itemize} #1 \end{itemize}}
\newcommand{\bfJ}{{\bf J}}
\newcommand{\bfI}{{\bf I}}
\newcommand{\bfx}{{\bf x}}
\newcommand{\bff}{{\bf f}}
\newcommand{\bfW}{{\bf W}}
\newcommand{\bfL}{{\bf L}}
\newcommand{\bfA}{{\bf A}}
\newcommand{\bfM}{{\bf M}}
\newcommand{\bfD}{{\bf D}}
\newcommand{\bfd}{{\bf d}}
\newcommand{\bfK}{{\bf K}}
\newcommand{\bfR}{{\bf R}}
\newcommand{\bfp}{{\bf p}}
\newcommand{\bfq}{{\bf q}}
\newcommand{\bfH}{{\bf H}}
\newcommand{\bfu}{{\bf u}}
\newcommand{\bfU}{{\bf U}}
\newcommand{\bfy}{{\bf y}}
\newcommand{\ttd}{{\tt d}}

%%-----------------------------------------------
\setbeamertemplate{navigation symbols}{}
\begin{document}
%%-----------------------------------------------
\title{程序设计基础与实验}
\author{刘新国}
\institute{浙江大学CAD\&CG国家重点实验室}
\date{\today}
%%-----------------------------------------------
\frame {
	\titlepage
}
%%-----------------------------------------------
\frame{
	\frametitle{课程介绍}
	\tableofcontents
}
%%-----------------------------------------------
\section{课程学习目标}
\frame { \frametitle{课程学习目标}
\items{
	\item C语言及其编程技术
	\item 基本的问题求解算法
	\item 理解高级程序设计语言的结构
	\item 掌握基本的程序设计过程和技巧
	\item 具备初步的高级语言程序设计能力。
	}
}
%%-----------------------------------------------
\section{课程主要内容}
\frame { \frametitle{课程主要内容}
\items{
	\item 数据基本类型与表达
	\item 程序基本流程控制
	\item 函数及程序模块化
	\item 数组与文件应用
	\item 算法基础,等等
	}
}
%%-----------------------------------------------
\section{参考教材}
\frame{ \frametitle{参考教材}
\items{
	\item 教材(国家精品课程主讲教材)\\
	\colorize{blue}{\large《C语言程序设计》(第三版)\\ 何钦铭 颜晖  主编 }
	\item 参考书 \\
	\colorize{blue}{\large 《C程序设计》\\ 谭浩强 编 }
}
}
%%-----------------------------------------------
\section{课件资源}
\frame { \frametitle{课件资源}
\items{
	\item 课件主页:\url{http://www.cad.zju.edu.cn/home/xgliu/C2016}
	\item 在线上机地址:\url{http://10.77.30.138}
	\item 教师答疑板:\url{http://www.cc98.org}
	\item 电话:13858115132
	\item 邮件:xgliu@cad.zju.edu.cn
	\item 办公室:蒙民伟楼524室,CAD\&CG国家重点实验室
	}
}
%%-----------------------------------------------
\section{课程评价考核}
\frame { \frametitle{评价考核}
\begin{columns} \begin{column}{0.35\textwidth}
\items {\item 理论知识部分
		\items { 
			\item 练习作业(5 \%)
			\item 期中测验(5 \%)
			\items{ \item 春学期最后一周
					\item 或者下学期第一周
					\item 随堂考试, 1个小时
			}
			\item 期末考试(50 \%)
			}	
		\item 上机实验部分
		\items { 
			\item 上机实验(15 \%)
			\item 上机考试(25 \%)
			}	
	}
\end{column}
\begin{column} {0.65\textwidth}
\items{ \item 平时表现部分
		\items { 
			\item 考勤、问问题、答问题、好程序
			\item 酌情加分
			}	
		\item 机考方法
		\items{
			\item 共4题。做出4题得25分,3题22分,2题19分,1题15分
			\item 得0分者,可以补考1次 \item 补考通过,机考成绩为15分
			\item \color<+->[rgb]{1,0,0} 
				  补考不通过,则本课程不及格
		}
		\item \colorize{blue}{可能进行微调,和其他班级保持一致}
	}
\end{column}
\end{columns}
}
%%-----------------------------------------------
\section{课程学习方式}
\frame { \frametitle{课程学习方式} 
\items {
	\item 上课、课后练习
	\item 上机在线练习(实验)
	\items{
		\item 第2周开始(第1周熟悉编程环境)
		\item 在线联系网址:(等待通知)
		\item 用户名\&口令都是学号\\(第一次登录之后可以修改密码)
		}
	\item 每周大约10 $\sim$ 20 题(第2周开始)
	\items {
		\item 一部分实验课当日结束之前完成
		\item 另一部分在5天内完成
		\item 以上机系统上设置的截止日期为准 
		\item \colorize{red}{\large 过期作业关闭,无法再做}
		}
	}
}
%====================================================
\end{document}

