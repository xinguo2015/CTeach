\documentclass{beamer}
\usepackage{ctex}
\usepackage{beamerthemetree}
\usepackage{graphicx}
\usepackage{listings}
\usepackage[autoplay,loop]{animate}
\usepackage{fancybox}
\usepackage{movie15}
\usepackage{verbatim}
%%-----------------------------------------------
%\usetheme{default}
\setbeamertemplate{navigation symbols}{}
\usetheme{AnnArbor}
\usecolortheme{lily}
%%-----------------------------------------------
\usepackage{tikz}
\usetikzlibrary{calc,positioning,shapes}
\usetikzlibrary{arrows,chains,decorations.pathreplacing}
\usetikzlibrary{shadows} % for using double copy shadow
%\usetikzlibrary{calc,positioning}
%\usetikzlibrary{shapes.geometric,shapes.symbols,shapes.misc}
\usetikzlibrary{shapes}
\usetikzlibrary{shadows} % for using double copy shadow
\usetikzlibrary{arrows,chains,decorations.pathreplacing}

\tikzset{
    start-end/.style={
        draw,
        rectangle,
        rounded corners,
    },
    input/.style={ % requires library shapes.geometric
        draw,
        trapezium,
        trapezium left angle=60,
        trapezium right angle=120,
    },
    operation/.style={
        draw,
        rectangle
    },
    loop/.style={ % requires library shapes.misc
        draw,
        chamfered rectangle,
        chamfered rectangle xsep=2cm
    },
    decision/.style={ % requires library shapes.geometric
        draw,
        diamond,
        aspect=#1
    },
    decision/.default=1,
    print/.style={ % requires library shapes.symbols
        draw,
        tape,
        tape bend top=none
    },
    connection/.style={
        draw,
        circle,
        radius=5pt,
    },
    process rectangle outer width/.initial=0.15cm,
    predefined process/.style={
        rectangle,
        draw,
        append after command={
        \pgfextra{
          \draw
          ($(\tikzlastnode.north west)-(0,0.5\pgflinewidth)$)--
          ($(\tikzlastnode.north west)-(\pgfkeysvalueof{/tikz/process rectangle outer width},0.5\pgflinewidth)$)--
          ($(\tikzlastnode.south west)+(-\pgfkeysvalueof{/tikz/process rectangle outer width},+0.5\pgflinewidth)$)--
          ($(\tikzlastnode.south west)+(0,0.5\pgflinewidth)$);
          \draw
          ($(\tikzlastnode.north east)-(0,0.5\pgflinewidth)$)--
          ($(\tikzlastnode.north east)+(\pgfkeysvalueof{/tikz/process rectangle outer width},-0.5\pgflinewidth)$)--
          ($(\tikzlastnode.south east)+(\pgfkeysvalueof{/tikz/process rectangle outer width},0.5\pgflinewidth)$)--
          ($(\tikzlastnode.south east)+(0,0.5\pgflinewidth)$);
        }  
        },
        text width=#1,
        align=center
    },
    predefined process/.default=1.75cm,
    man op/.style={ % requires library shapes.geometric
        draw,
        trapezium,
        shape border rotate=180,
        text width=2cm,
        align=center,
    },
    extract/.style={
        draw,
        isosceles triangle,
        isosceles triangle apex angle=60,
        shape border rotate=90
    },
    merge/.style={
        draw,
        isosceles triangle,
        isosceles triangle apex angle=60,
        shape border rotate=-90
    },
	multidocument/.style={
		shape=tape,
		draw,
		fill=white,
		tape bend top=none,
		double copy shadow
	},
	manual input/.style={
		shape=trapezium,
		draw,
		shape border rotate=90,
		trapezium left angle=90,
		trapezium right angle=80
	}
}


\definecolor{mygreen}{rgb}{0,0.6,0}
\definecolor{mygray}{rgb}{0.5,0.5,0.5}
\definecolor{mymauve}{rgb}{0.58,0,0.82}
\lstdefinestyle{customc}{ language=C,
	backgroundcolor=\color{white},   % choose the background color
	basicstyle=\small\ttfamily,% size of fonts used for the code
	belowcaptionskip=1\baselineskip,
	breaklines=true,                 % automatic line breaking only at whitespace
	captionpos=b,                    % sets the caption-position to bottom
	commentstyle=\color{mygreen},    % comment style
	escapeinside={\%*}{*)},          % if you want to add LaTeX within your code
	frame=L,
	frame=shadowbox, 
	%framexleftmargin=5mm, 
	numbers=none,%left,
	commentstyle=\itshape\color{purple!40!black},
	rulesepcolor=\color{red!20!green!20!blue!20!},
	identifierstyle=\color{black},
	keywordstyle=\bfseries\color{blue}, % keyword style
	stringstyle=\color{red},% string literal style
	tabsize=4,
	xleftmargin=\parindent
}

\lstdefinestyle{basic}{
	backgroundcolor=\color{white},   % choose the background color; you must add \usepackage{color} or \usepackage{xcolor}
	%basicstyle=\footnotesize,        % the size of the fonts that are used for the code
	basicstyle=\small,        % the size of the fonts that are used for the code
	breakatwhitespace=false,         % sets if automatic breaks should only happen at whitespace
	breaklines=true,                 % sets automatic line breaking
	captionpos=b,                    % sets the caption-position to bottom
	commentstyle=\color{mygreen},    % comment style
	deletekeywords={},               % if you want to delete keywords from the given language
	escapeinside={\%*}{*)},          % if you want to add LaTeX within your code
	extendedchars=true,              % lets you use non-ASCII characters; for 8-bits encodings only, does not work with UTF-8
	frame=single,	                 % adds a frame around the code
	keepspaces=true,                 % keeps spaces in text, useful for keeping indentation of code (possibly needs columns=flexible)
	keywordstyle=\color{blue},       % keyword style
	otherkeywords={*,...},           % if you want to add more keywords to the set
	numbers=left,                    % where to put the line-numbers; possible values are (none, left, right)
	numbersep=5pt,                   % how far the line-numbers are from the code
	numberstyle=\tiny\color{mygray}, % the style that is used for the line-numbers
	rulecolor=\color{black},         % if not set, the frame-color may be changed on line-breaks within not-black text (e.g. comments (green here))
	showspaces=false,                % show spaces everywhere adding particular underscores; it overrides 'showstringspaces'
	showstringspaces=false,          % underline spaces within strings only
	showtabs=false,                  % show tabs within strings adding particular underscores
	stepnumber=1,                    % the step between two line-numbers. If it's 1, each line will be numbered
	stringstyle=\color{mymauve},     % string literal style
	tabsize=3,	                     % sets default tabsize to 2 spaces
	%title=\lstname                   % show the filename of files included with \lstinputlisting; also try caption instead of title
}

%%-----------------------------------------------
\newcommand{\mypause}{\pause}
%\newcommand{\mypause}{} % for no pause
\newcommand{\minorpause}{\pause}
\newcommand{\colorize}[2]{{\color{#1} #2 }}
\newcommand{\items}[1]{\begin{itemize} #1 \end{itemize}}
\newcommand{\bfy}{{\bf y}}
\newcommand{\ttd}{{\tt d}}
%%-----------------------------------------------
\begin{document}
%%-----------------------------------------------
\title{程序设计基础与实验}
\author{刘新国}
\institute{浙江大学CAD\&CG国家重点实验室}
\date{\today}
%%-----------------------------------------------
\frame {
	\titlepage
}
%%-----------------------------------------------
\frame{ \frametitle{第一章~~引言}
	\tableofcontents
}
%%-----------------------------------------------
\section{本章要点}
\frame { \frametitle{本章要点}
\items {
	\item 什么是程序?程序设计语言包含哪些功能?
	\item 程序设计语言在语法上包含哪些内容?
	\item 结构化程序设计有哪些基本的控制结构?
	\item C语言有哪些特点?
	\item C语言程序的基本框架如何?
	\item 形成一个可运行的C语言程序主要步骤?
	\item 如何用流程图描述简单的算法?
	}
}
%%-----------------------------------------------
\section{一个C语言程序}
\frame { %\frametitle{一个C语言程序}
	\lstinputlisting[style=customc, language=C, firstline=1,
	] {code/1-1.c}

}
\frame { \frametitle{C语言程序的构成}
\items {
	\item 一些函数:factorial, main, scanf, printf \items { 
		\item factorial和main是自己设计的函数
		\item scanf和printf是系统提供的函数
		}
	\item 函数的构成\items {
		\item 一些变量:i, fact, 等等
		\item 一些语句:流程控制,函数调用,计算赋值,等等
	}
}
\minorpause
\center \colorize{red}{ \large 程序从main函数开始执行}
}

%%-----------------------------------------------
\section{程序与程序设计语言}
\frame { \frametitle{程序与程序设计语言}
	\begin{columns}[T]
	\column{0.5\textwidth}
	\items {
		\item 程序可看作是\colorize{red}{一系列加工步骤}\items{
			\item 一系列\colorize{red}{计算机指令}的有序组合
			\item 指示计算机对数据进行处理,解决实际问题
			}
		\item 计算机指令 \items{
			\item 执行一个最基本的功能
			\item 算术运算:加减乘除,比较大小等等
			\item 输入输出,控制指令等等
			}
		}
	\column{0.5\textwidth}
	\items {
		\item 指令系统 \items{
			\item 计算机所能实现的指令集合
			\item 不同的计算机有不同指令系统 
			}
		\item 编写指令程序 \items{
			\item 繁琐、效率低
			\item 可读性差、不宜维护
			\item 与指令系统相关、难以移植
			\item 需要高级程序设计语言
			}
		}
	\end{columns}
}

%%-----------------------------------------------
\frame { \frametitle{程序设计语言}
	\colorize{red}{可看作是编写程序的一种工具}
	\begin{columns}[T]
	\column{0.6\textwidth}
	\items {
		\item 程序设计语言的功能 \items{
			\item 数据表达(data representation)\items {
				\item 基本类型:整数、浮点数、字符
				\item 复合类型(用户自定义的)
				\item 常量、变量
				}
			\item 流程控制(flow control)\items{
				\item 顺序结构(Sequential Control Structure)
				\item 分支结构(Branch Control Structure)
				\item 循环结构(Loop Control Structure)
				}
			}
		}
	\column{0.6\textwidth}
	\items{
		\item 程序设计语言的语法\items{
			\item 单词
			\item 表达式
			\item 语句
			}
		}
	\end{columns}
}
%%-----------------------------------------------
\frame { \frametitle{C语言的基本语法}
\items{
	\item C语言的单词\items{
		\item 保留字:\colorize{red}{C语言规定的、赋予它们以特定含义、有专门用途的标识符}。例如:int, float, char, if, else, for, while, ...等等。
		\item 用户定义的标识符:\colorize{red}{由字母、数字以及下划线组成,且第一个字符必须是字母或下划线}。例如:factorial
		\item 常量:常量具有数据类型,包括\colorize{red}{整数常量,浮点数常量,字符常量,字符串常量},等等。例如: 123,5.4,'A',"hello world"。
		\item 运算符:表示\colorize{red}{对数据所进行一种运算}。例如:$+ ~ - ~ * ~ / ~ \% ~ > ~ < ~ >= ~ <= ~ == ~ = ~ $,等等
		}
	}
}
%%-----------------------------------------------
\frame { \frametitle{C语言的语法}
\items{
	\item C语言的语法单位\items{
		\item 变量定义,例如 int n; 定义了一个整数变量n。
		\item 表达式:运算对象和运算符的有序组合。运算对象可以常量或者变量。例如: 2+3*n,n+2<5等等
		\item 语句:最基本的执行单位。程序的功能通过执行一系列的语句来实现。
		\items{
			\item 表达式语句
			\item 分支语句
			\item 循环语句
			\item 符合语句
			\item 函数定义语句
			\item 函数调用语句
			}
		}
	}
}
%%-----------------------------------------------
\frame { \frametitle{C语言的特点}
\items{
	\item 一种结构化语言
	\item 语句简洁、紧凑,使用方便、灵活
	\item 易于移植:不包含与硬件有关的因素
	\item 有强大的处理能力
	\item 目标代码运行效率高
	\item 数据类型检查不严格
	\item 区分大小写
	}
}
%%-----------------------------------------------
\section { 程序设计的一般过程}
\frame { \frametitle{程序设计的一般过程}
	% Start the picture
	\center
	\begin{tikzpicture}[%
			>=triangle 60,              % Nice arrows; your taste may be different
			start chain=going below,    % General flow is top-to-bottom
			node distance=6mm and 60mm, % Global setup of box spacing
			every join/.style={norm},   % Default linetype for connecting boxes
		]
		% ------------------------------------------------- 
		% A few box styles 
		% <on chain> *and* <on grid> reduce the need for manual relative
		% positioning of nodes
		\tikzset{
			% 定义基本形状
			base/.style={draw, on chain, align=center, minimum height=4ex},
			terminal/.style={base, rounded corners},
			process/.style={base, rectangle, text width=6em},
			decision/.style={base, diamond, aspect=2, text width=5em},
			multidocument/.style={ shape=tape, draw, fill=white, tape bend top=none, 
				double copy shadow },
			% coord node style is used for placing corners of connecting lines
			coord/.style={coordinate, on chain, on grid, node distance=6mm and 25mm},
			% nmark node style is used for coordinate debugging marks
			nmark/.style={draw, cyan, circle, font={\sffamily\bfseries}},
			% -------------------------------------------------
			% Connector line styles for different parts of the diagram
			norm/.style={->, draw, red},
			free/.style={->, draw, blue},
			cong/.style={->, draw, green},
			it/.style={font={\small\itshape}}
		}
		% -------------------------------------------------
		% Start by placing the nodes
		\node [terminal]      {开始};
		\node [process, join] (edit) {编辑程序};
		\node [process, join] (compile) {编译程序};
		\node [process, join] (link) {链接程序};
		\node [process, join] (run) {运行测试};
		\node [terminal, join] {结束};
		% mark some coordinate for connecting
		\node [coord, left=4cm of edit] (c1) {};
		\node [coord, left=4cm of compile] (c2) {};
		\node [coord, left=4cm of link] (c3) {};
		\node [coord, left=4cm of run] (c4) {};
		% output documents
		\mypause
		\node [multidocument, right =1cm of edit, text width=8em] (code) {源文件(.c/.h)};
		\draw [->, green] (edit) -- (code);
		\node [multidocument, right =1cm of compile, text width=8em] (obj) {目标文件(.obj)};
		\draw [->, green] (compile) -- (obj);
		\node [multidocument, right =1cm of link, text width=8em] (exe) {可执行文件};
		\draw [->, green] (link) -- (exe);
		% error edit
		\mypause
		\draw [->, blue] (compile)--(c2);
		\path (compile) to node [fill=white]{编译出错} (c2);
		\draw [->, blue] (link)--(c3);
		\path (link) to node [fill=white]{链接出错} (c3);
		\draw [->, blue] (run)--(c4);
		\path (run) to node [fill=white]{运行出错} (c4);
		\draw [->, blue] (c4)|-(edit);
	\end{tikzpicture}
}

%%-----------------------------------------------
%%-----------------------------------------------

\end{document}
