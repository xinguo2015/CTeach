\documentclass{beamer}
\usepackage{ctex}
\usepackage{beamerthemetree}
\usepackage{graphicx}
\usepackage{listings}
\usepackage[autoplay,loop]{animate}
\usepackage{fancybox}
\usepackage{movie15}
\usepackage{verbatim}
%%-----------------------------------------------
%\usetheme{default}
\setbeamertemplate{navigation symbols}{}
\usetheme{AnnArbor}
\usecolortheme{lily}
%%-----------------------------------------------
\usepackage{tikz}
\usetikzlibrary{calc,positioning,shapes}
\usetikzlibrary{arrows,chains,decorations.pathreplacing}
\usetikzlibrary{shadows} % for using double copy shadow
%\usetikzlibrary{calc,positioning}
%\usetikzlibrary{shapes.geometric,shapes.symbols,shapes.misc}
\usetikzlibrary{shapes}
\usetikzlibrary{shadows} % for using double copy shadow
\usetikzlibrary{arrows,chains,decorations.pathreplacing}

\tikzset{
    start-end/.style={
        draw,
        rectangle,
        rounded corners,
    },
    input/.style={ % requires library shapes.geometric
        draw,
        trapezium,
        trapezium left angle=60,
        trapezium right angle=120,
    },
    operation/.style={
        draw,
        rectangle
    },
    loop/.style={ % requires library shapes.misc
        draw,
        chamfered rectangle,
        chamfered rectangle xsep=2cm
    },
    decision/.style={ % requires library shapes.geometric
        draw,
        diamond,
        aspect=#1
    },
    decision/.default=1,
    print/.style={ % requires library shapes.symbols
        draw,
        tape,
        tape bend top=none
    },
    connection/.style={
        draw,
        circle,
        radius=5pt,
    },
    process rectangle outer width/.initial=0.15cm,
    predefined process/.style={
        rectangle,
        draw,
        append after command={
        \pgfextra{
          \draw
          ($(\tikzlastnode.north west)-(0,0.5\pgflinewidth)$)--
          ($(\tikzlastnode.north west)-(\pgfkeysvalueof{/tikz/process rectangle outer width},0.5\pgflinewidth)$)--
          ($(\tikzlastnode.south west)+(-\pgfkeysvalueof{/tikz/process rectangle outer width},+0.5\pgflinewidth)$)--
          ($(\tikzlastnode.south west)+(0,0.5\pgflinewidth)$);
          \draw
          ($(\tikzlastnode.north east)-(0,0.5\pgflinewidth)$)--
          ($(\tikzlastnode.north east)+(\pgfkeysvalueof{/tikz/process rectangle outer width},-0.5\pgflinewidth)$)--
          ($(\tikzlastnode.south east)+(\pgfkeysvalueof{/tikz/process rectangle outer width},0.5\pgflinewidth)$)--
          ($(\tikzlastnode.south east)+(0,0.5\pgflinewidth)$);
        }  
        },
        text width=#1,
        align=center
    },
    predefined process/.default=1.75cm,
    man op/.style={ % requires library shapes.geometric
        draw,
        trapezium,
        shape border rotate=180,
        text width=2cm,
        align=center,
    },
    extract/.style={
        draw,
        isosceles triangle,
        isosceles triangle apex angle=60,
        shape border rotate=90
    },
    merge/.style={
        draw,
        isosceles triangle,
        isosceles triangle apex angle=60,
        shape border rotate=-90
    },
	multidocument/.style={
		shape=tape,
		draw,
		fill=white,
		tape bend top=none,
		double copy shadow
	},
	manual input/.style={
		shape=trapezium,
		draw,
		shape border rotate=90,
		trapezium left angle=90,
		trapezium right angle=80
	}
}


%%-----------------------------------------------
\newcommand{\colorize}[2]{{\color{#1} #2 }}
\newcommand{\items}[1]{\begin{itemize} #1 \end{itemize}}
\newcommand{\bfy}{{\bf y}}
\newcommand{\ttd}{{\tt d}}
%%-----------------------------------------------
\begin{document}
%%-----------------------------------------------
\title{程序设计基础与实验}
\author{刘新国}
\institute{浙江大学CAD\&CG国家重点实验室}
\date{\today}
%%-----------------------------------------------
\frame {
	\titlepage
}
%%-----------------------------------------------
\frame{
	\frametitle{引言}
	\tableofcontents
}
%%-----------------------------------------------
\section{C语言介绍}
\frame { 
	\frametitle{程序设计过程}
	% Start the picture
	\center
	\begin{tikzpicture}[%
			>=triangle 60,              % Nice arrows; your taste may be different
			start chain=going below,    % General flow is top-to-bottom
			node distance=6mm and 60mm, % Global setup of box spacing
			every join/.style={norm},   % Default linetype for connecting boxes
		]
		% ------------------------------------------------- 
		% A few box styles 
		% <on chain> *and* <on grid> reduce the need for manual relative
		% positioning of nodes
		\tikzset{
			% 定义基本形状
			base/.style={draw, on chain, align=center, minimum height=4ex},
			terminal/.style={base, rounded corners},
			process/.style={base, rectangle, text width=6em},
			decision/.style={base, diamond, aspect=2, text width=5em},
			multidocument/.style={ shape=tape, draw, fill=white, tape bend top=none, 
				double copy shadow },
			% coord node style is used for placing corners of connecting lines
			coord/.style={coordinate, on chain, on grid, node distance=6mm and 25mm},
			% nmark node style is used for coordinate debugging marks
			nmark/.style={draw, cyan, circle, font={\sffamily\bfseries}},
			% -------------------------------------------------
			% Connector line styles for different parts of the diagram
			norm/.style={->, draw, red},
			free/.style={->, draw, blue},
			cong/.style={->, draw, green},
			it/.style={font={\small\itshape}}
		}
		% -------------------------------------------------
		% Start by placing the nodes
		\node [terminal]      {开始};
		\node [process, join] (edit) {编辑程序};
		\node [process, join] (compile) {编译程序};
		\node [process, join] (link) {链接程序};
		\node [process, join] (run) {运行测试};
		\node [terminal, join] {结束};
		% mark some coordinate for connecting
		\node [coord, left=4cm of edit] (c1) {};
		\node [coord, left=4cm of compile] (c2) {};
		\node [coord, left=4cm of link] (c3) {};
		\node [coord, left=4cm of run] (c4) {};
		% output documents
		\pause
		\node [multidocument, right =1cm of edit, text width=8em] (code) {源文件(.c/.h)};
		\draw [->, green] (edit) -- (code);
		\node [multidocument, right =1cm of compile, text width=8em] (obj) {目标文件(.obj)};
		\draw [->, green] (compile) -- (obj);
		\node [multidocument, right =1cm of link, text width=8em] (exe) {可执行文件};
		\draw [->, green] (link) -- (exe);
		% error edit
		\pause
		\draw [->, blue] (compile)--(c2);
		\path (compile) to node [fill=white]{编译出错} (c2);
		\draw [->, blue] (link)--(c3);
		\path (link) to node [fill=white]{链接出错} (c3);
		\draw [->, blue] (run)--(c4);
		\path (run) to node [fill=white]{运行出错} (c4);
		\draw [->, blue] (c4)|-(edit);
	\end{tikzpicture}
}

%%-----------------------------------------------

\frame{
	\lstinputlisting[ 
		language=C, 
		basicstyle=\small\ttfamily,
		numbers=none,%left,
		keywordstyle=\color{blue},
		numbersep=5pt,
		commentstyle=\small\color{green!20!blue!80!},
		showspaces=false,
		%showtabs=true,
		tabsize=4,
		frame=shadowbox, 
		framexleftmargin=5mm, 
		rulesepcolor=\color{red!20!green!20!blue!20!},
		firstline=1] {code/1-1.c}
}
%%-----------------------------------------------

\end{document}
